% arara: xelatex
\documentclass[12pt]{article}

% \usepackage{physics}

\usepackage{hyperref}
\hypersetup{
    colorlinks=true,
    linkcolor=blue,
    filecolor=magenta,      
    urlcolor=cyan,
    pdftitle={Overleaf Example},
    pdfpagemode=FullScreen,
    }

\usepackage{tikzducks}

\usepackage{tikz} % картинки в tikz
\usepackage{microtype} % свешивание пунктуации

\usepackage{array} % для столбцов фиксированной ширины

\usepackage{indentfirst} % отступ в первом параграфе

\usepackage{sectsty} % для центрирования названий частей
\allsectionsfont{\centering}

\usepackage{amsmath, amsfonts, amssymb} % куча стандартных математических плюшек

\usepackage{mathtools}
\usepackage{comment}

\usepackage[top=2cm, left=1.2cm, right=1.2cm, bottom=2cm]{geometry} % размер текста на странице

\usepackage{lastpage} % чтобы узнать номер последней страницы

\usepackage{enumitem} % дополнительные плюшки для списков
%  например \begin{enumerate}[resume] позволяет продолжить нумерацию в новом списке
\usepackage{caption}

\usepackage{url} % to use \url{link to web}


\newcommand{\smallduck}{\begin{tikzpicture}[scale=0.3]
    \duck[
        cape=black,
        hat=black,
        mask=black
    ]
    \end{tikzpicture}}

\usepackage{fancyhdr} % весёлые колонтитулы
\pagestyle{fancy}
\lhead{}
\chead{}
\rhead{Econometrics midterm demo, March 2025}
\lfoot{}
\cfoot{}
\rfoot{}

\renewcommand{\headrulewidth}{0.4pt}
\renewcommand{\footrulewidth}{0.4pt}

\usepackage{tcolorbox} % рамочки!

\usepackage{todonotes} % для вставки в документ заметок о том, что осталось сделать
% \todo{Здесь надо коэффициенты исправить}
% \missingfigure{Здесь будет Последний день Помпеи}
% \listoftodos - печатает все поставленные \todo'шки


% более красивые таблицы
\usepackage{booktabs}
% заповеди из докупентации:
% 1. Не используйте вертикальные линни
% 2. Не используйте двойные линии
% 3. Единицы измерения - в шапку таблицы
% 4. Не сокращайте .1 вместо 0.1
% 5. Повторяющееся значение повторяйте, а не говорите "то же"


\setcounter{MaxMatrixCols}{20}
% by crazy default pmatrix supports only 10 cols :)


\usepackage{fontspec}
\usepackage{libertine}
\usepackage{polyglossia}

\setmainlanguage{english}
\setotherlanguages{english}

% download "Linux Libertine" fonts:
% http://www.linuxlibertine.org/index.php?id=91&L=1
% \setmainfont{Linux Libertine O} % or Helvetica, Arial, Cambria
% why do we need \newfontfamily:
% http://tex.stackexchange.com/questions/91507/
% \newfontfamily{\cyrillicfonttt}{Linux Libertine O}

\AddEnumerateCounter{\asbuk}{\russian@alph}{щ} % для списков с русскими буквами
% \setlist[enumerate, 2]{label=\asbuk*),ref=\asbuk*}

%% эконометрические сокращения
\DeclareMathOperator{\rank}{rank}
\DeclareMathOperator{\Cov}{\mathbb{C}ov}
\DeclareMathOperator{\Corr}{\mathbb{C}orr}
\DeclareMathOperator{\Var}{\mathbb{V}ar}
\DeclareMathOperator{\pCorr}{\mathrm{pCorr}}
\DeclareMathOperator{\col}{col}
\DeclareMathOperator{\row}{row}
\DeclareMathOperator{\hCov}{\widehat{Cov}}
\DeclareMathOperator{\hVar}{\widehat{Var}}


\let\P\relax
\DeclareMathOperator{\P}{\mathbb{P}}

\DeclarePairedDelimiter{\abs}{\lvert}{\rvert}
\DeclarePairedDelimiter{\norm}{\lVert}{\rVert}

\let\H\relax
\DeclareMathOperator{\H}{\mathbb{H}}


\DeclareMathOperator{\E}{\mathbb{E}}
% \DeclareMathOperator{\tr}{trace}
\DeclareMathOperator{\card}{card}

\DeclareMathOperator{\Convex}{Convex}

\newcommand{\SSR}{SS^{\text{res}}}
\newcommand{\SSE}{SS^{\text{expl}}}
\newcommand{\SST}{SST}
\newcommand{\hb}{\hat\beta}
\newcommand{\hy}{\hat y}
\newcommand \cN{\mathcal{N}}
\newcommand \dN{\mathcal{N}}
\newcommand \dBin{\mathrm{Bin}}


\newcommand \RR{\mathbb{R}}
\newcommand \NN{\mathbb{N}}





\begin{document}

\section*{General notes}

Midterm will contain 6 problems. 
All the problems have equal weights. 
Midterm duration is 120 minutes. 
Closed book, one A4 cheatsheet is allowed.

\section*{Demo variant «Dolly»}
\begin{enumerate}
    \item Let $X$ be a random variable that follows a uniform distribution on the interval $[10,20]$ and 
    $g(X) = \frac{1}{X}$.
    
    \begin{enumerate}
        \item Find $\E(g(X))$ and $\Var(g(X))$ exactly by computing the expectation and variance directly from the definition of expectation.
        
        \item Use the delta method to approximate $\E(g(X))$ and $\Var(g(X))$.
        
        \item Compare the exact variance obtained in part (a) with the approximation from part (b). 
        Discuss the accuracy of the delta method in this case.
    \end{enumerate}
    \item Consider a logistic regression model for the probability of success in the Econometrics course:
    \[
    \P(y = 1 \mid h) = \frac{\exp(\beta_0 + \beta_1 h)}{1 + \exp(\beta_0 + \beta_1 h)},
    \]
    where $y$ is a binary outcome variable, and $h$ is a number of drawn hedgehogs.
    
    Suppose that the model has been estimated using a dataset of $n = 1000$ students, 
    and the estimated coefficients along with their standard errors are:
    $\hb_0 = -2.5$ with $se(\hb_0) = 0.5$ and  $\hb_1 = 1.2$ with $se(\hb_1) = 0.3$.
    
    \begin{enumerate}
        \item Compute the predicted probability $\hat{p}$ for $h = 2$.
    
        \item Use the delta method to approximate the variance of $\hat{p}$ for $h = 2$.
    
        \item Construct an approximate 95\% confidence interval for $p$ using a normal approximation and delta method.
    
        \item Discuss the limitations of using the delta method in this context.
    \end{enumerate}
    \item The dataset contains $1000$ observations. 
    We have estimated logistic regression A:
    \[
    \hat\P(y_i = 1 \mid a_i, b_i, c_i, d_i) = \Lambda(0.3 + 0.1 a_i + 0.2 b_i - 0.3 c_i + 0.2 d_i), \quad \ln L = -330,
    \]
    and logistic regression B:
    \[
    \hat\P(y_i = 1 \mid a_i, b_i, c_i, d_i) = \Lambda(0.1 + 0.4 a_i), \quad \ln L = -335.
    \]
    Here $\ln L$ denotes the maximal value of the log-likelihood function. 
    \begin{enumerate}
        \item Compare these two nested models using $LR$-test. 
        Use 5\% significance level.
        Clearly state $H_0$, $H_a$, the distribution of the test statistic under $H_0$ and critical region.
        \item Compare these models using corrected Akaike information criterion. 
        \item Calculate $\ln L$ for the trivial model $\hat\P(y_i = 1 \mid a_i, b_i, c_i, d_i) = \Lambda(\hb_0)$ given that $y_i = 1$ in $30\%$ of the observations.
    \end{enumerate}


    \item The dataset contains $1000$ observations. 
    Winnie-the-Pooh has estimated logistic model for the probability that honey is good ($h_i = 1$).
    Predictor $x_i$ is the height of the tree and $b_i = 1$ for good bees and $b_i = 0$ for bad bees.
    \[
    \hat \P(h_i = 1 \mid x_i, b_i ) = \Lambda(0.2 + 0.1 x_i + 0.2 b_i).
    \]
    \begin{enumerate}
        \item For which height of the tree the marginal effect $\partial \hat P/\partial x$ is maximal given that bees are good?
        \item Draw the region of the plane where the predicted probability is from $0.2$ to $0.4$.
        \item For which height of the tree the partial effect of bees type (good bees minus bad bees) is maximal?
    \end{enumerate}

    \item Consider the simultaneous equation model with endogeneous variables $v_i$ and $w_i$.
    \[
    \begin{cases}
        v_i = \alpha_1 + \alpha_2 x_i + \alpha_3 w_i + \alpha_4 a_i + \alpha_5 b_i + u_{1i} \\
        w_i = \beta_1 + \beta_2 v_i + \beta_3 b_i + u_{2i}.
    \end{cases}
    \]

    \begin{enumerate}
        \item Check the order identification condition for each equation. 
        \item Check the rank identification condition for each equation. 
    \end{enumerate}

    \item Something on logit model from LSE external exam. 
\end{enumerate}

\end{document}
\section*{Demo variant «Sailor Moon»}
\begin{enumerate}
    \item 
    \item 
    \item 
    \item 
    \item 

    \item Something on logit model from LSE external exam. 

\end{enumerate}





\end{document}

